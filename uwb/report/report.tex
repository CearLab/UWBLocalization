\documentclass[a4paper, 12pt, titlepage]{article}
\usepackage[utf8]{inputenc}   % unicode package
\usepackage[english]{babel}   % language package
\usepackage{cancel}           % math style package
\usepackage{bm} 			  % text style package
\usepackage{mathtools}        % math style package
\usepackage{amsmath}          % math style package
\usepackage{enumitem}
\usepackage{amsfonts}
\usepackage{systeme}          % math style package
\usepackage{color}            % text style package
\usepackage{subfigure}
\usepackage{fancyvrb}
\usepackage{float} 			  % image style package
\usepackage{tikz}
\usetikzlibrary{arrows.meta, automata}
\usepackage[margin=1in]{geometry}

\linespread{1.2}

%% Equation number like the section's
\numberwithin{equation}{section}
\numberwithin{figure}{section}

\date{\Large{\today}}
\title{\huge{Utilizzo dei moduli UWB}}
\begin{document}
\maketitle
\setlength{\abovedisplayskip}{3mm}
\setlength{\belowdisplayskip}{3mm}
\newpage

\section{Introduzione}
    Questo documento spiega brevemente come utilizzare i moduli \textbf{UltraWide Band (UWB)} per lo scambio di dati e il misuramento delle distanza relative tra due o più agenti.
\section{Struttura}
    I componenti necessari al corretto utilizzo dei moduli UWB sono due:
    \begin{itemize}
        \item{un processo che si occupa della comunicazione tra i moduli}
        \item{delle API per la configurazione, scrittura e lettura dei dati}
    \end{itemize}
    Per poter utilizzare i moduli UWB è necessario lanciare preventivamente il comando
    \vspace{0.5em}
    \begin{figure}[h]
        \begin{BVerbatim}
                ./ranging_service <UWB device file>
        \end{BVerbatim}
    \end{figure}
    dove il secondo argomento rappresenta il device file creato una volta collegato il modulo UWB tramite cavo USB (di solito è qualcosa tipo \textit{/dev/ttyUSB0}, \textit{/dev/ttyAMA0}, etc.). \\ [1em]
    \textit{nota:} si consiglia di lanciare il comando
    \vspace{0.5em}
    \begin{figure}[h]
        \begin{BVerbatim}
            nohup ./ranging_service <UWB device file> &
        \end{BVerbatim}
    \end{figure}
    \noindent
    \newline
    in modo da avere il programma in background e svincolarlo dal terminale in uso. \\[2em]
    \noindent
    Una volta fatto ciò si può passare ad utilizzare le API messe a disposizione nel file \textit{source/uwb\_mwc.c}:
    \begin{itemize}
        \item{inizializzare i moduli tramite la funzione \texttt{initialize\_UWB()}}
        \item{trasmettere dei dati tramite la funzione \texttt{write\_UWB(double data\_1, double data\_2)}}
        \item{leggere i dati tramite la funzione \texttt{read\_UWB()}}
        \item{recuperare i dati letti tramite la funzione \texttt{get\_UWB\_data()}}
    \end{itemize}
    \vspace{2em}
    \newpage
\section{Specifiche}
    Di seguito sono specificate alcuni dettagli sui tipi di dati e su accortezze da avere per usare correttamente i moduli UWB.

    \begin{itemize}
        \item{il tempo di trasmissione dei dati e della funzionalità di \textit{ranging} (i.e. il calcolo delle distanze relative) impiega circa \textbf{500ms}, quindi è consigiabile chiamare le funzioni di scrittura/lettura con una freuenza non superiore, altrimenti potreste ancora leggere i vecchi valori}
    \item{i dati vengono scambiati in modalità \textit{broadcast} quindi tutti gli agenti riceveranno i dati di tutti gli altri agenti}
    \item{è \textbf{necessario} che il modulo UWB con id \texttt{0x4829} sia attivo e funzionante, in quanto è quello che gestisce la comunicazione tra tutti i moduli UWB}
    \item{le distanze misurate (contenute nella variabile \texttt{uint16\_t distances[6]}) sono espresse in \textbf{centimetri (cm)}}
    \item{nel caso in cui venga letta una distanza di \texttt{65535} può significare o che c'è stato un problema durante la funzione di ranging o che qualche nodo non è attivo: in entrambi i casi la funzione \texttt{read\_UWB()} ritornerà il valore \texttt{FALSE}}
    \end{itemize}
        
\end{document}
