%% intro section
\section{Problem statement}
\label{probstat}

The main goal of this work consists in proposing a localization system based on a hybrid observer and fusing ranging measurents with an IMU sensor. The proposed theoretical framework extends the results from \cite{Oliva03} to estimate the orientation as well. Furthermore, the proposed solution will be tested on a Jackal rover from Clearpath. As reported in \ref{intro}, we aim to reach a good localization precision in a 3D GPS-denied environment, where the agent (i.e., the rover) is not limited to patrolling, but is also exploited for environment shaping and earth movementation tasks.

\medskip
Before delving into the details of the proposed solution, we now proceed by describing the setup considered, as well as the fundamental of the considered theoretical framework.

\medskip
This work builds on top of the preliminary results obtained in \cite{Oliva03}, where a hybrid observer for absolute 3D position estimate was developed for a rover, considering only acceleration and ranging measurements from a set of known anchors. More specifically, the need for a hybrid observer came from the multirate acquiring of the acceleration and distance measurements. Indeed, the proposed solution aimed at improving the position estimate performances of a standard multirate EKF, particularly on the z-axis. The poor performance of the EKF on the z-axis was caused by the nonlinearities of the system output mapping with respect to that particular direction. The observer proposed in \cite{Oliva03} combines the filtered IMU dynamics with a low-frequency linear jump map depending on the ranging measurements for the related innovation term.

\medskip
This work aims at extending this result to estimate the orientation of the rover, by considering acceleration and angular velocity measurements of the center-of-mass of the rover, and also ranging distances of three non-collinear tags from a set of $N=4$ fixed and known anchors.

\subsection{System dynamics}
\label{probstat:dynamics}

The model dynamics for the rover are described by the following equations:

\begin{subequations}
	\begin{align}
		 & \mathcal{P} \ : \
		\begin{cases}
			\dot{\bm{p}}          & = \bm{v}                                                   \\
			\dot{\bm{v}}          & = \bm{u}_a                                                 \\
			\dot{\bm{b}}          & = 0,                                                       \\
			\dot{\bm{q}}          & = \dfrac{1}{2}\bm{\Omega(\omega - \bm{b_{\omega}})}\bm{q}, \\
			\dot{\bm{b}}_{\omega} & = 0
		\end{cases}
		\bm{\longrightarrow}
		\quad\dot{\bm{x}} = A\bm{x} + B\bm{u} , \label{probstat:eqn:plant_dyn} \\
		 & \qquad \qquad \ \bm{y} =
		\begin{bmatrix}
			\bm{d} \\
			\bm{a} \\
			\bm{\omega}
		\end{bmatrix}
		=
		\begin{bmatrix}
			\bm{d}^{\star} \\
			\bm{u}_a       \\
			\bm{u}_{\omega}
		\end{bmatrix}
		+
		\begin{bmatrix}
			\bm{\nu}_d          \\
			\bm{b} + \bm{\nu}_a \\
			\bm{b}_{\omega} + \bm{\nu}_{\omega}
		\end{bmatrix},
		\label{probstat:eqn:plant_dyn_output}
	\end{align}
\end{subequations}

\medskip
where $\bm{x} = [\bm{p}^T \bm{v}^T \bm{b}^T \bm{\omega}^T \bm{b}_{\omega}^T]^T \in \mathbb{R}^{16}$ is the plant state vector, $\bm{p},\bm{v},\bm{\omega} \in \mathbb{R}^3$ the rover center-of-mass position, linear and angular velocity, $\bm{u} = [\bm{u}_a^T,\bm{u}_{\omega}^T]^T \in \mathbb{R}^6$ the unknown inputs of the system, $\bm{q}\in\mathbb{R}^4$ the quaternion expressing the orientation, and $\bm{b},\bm{b}_{\omega}\in\mathbb{R}^3$ the biases on the acceleration and angular velocity measurements, respectively. Indeed, both biases also enter the model output vector $\bm{y}\in\mathbb{R}^{3\cdot N + 6}$, along with the Gaussian noises $\bm{\nu}_d\in\mathbb{R}^{3\cdot N}$, and $\bm{\nu}_a,\bm{\nu}_{\omega}\in\mathbb{R}^3$. The noise characteristics will be specified later in the paper, when the actual observer design and performance assessment will be carried out. Note that the measurement vector $\bm{y}$ considers the ranging distances of three non-collinear tags disposed on the rover with respect to $N$ fixed and known anchors, together with linear acceleration $\bm{a}$ and angular velocity $\bm{\omega}$. More specificallly, we define the ranging distance true measurement as $d_{ij}^{\star} = \bm{p}_{A,i} - \bm{p}_{T,j}$, where $\bm{p}_{A,i}\in\mathbb{R}^3$ is the known absolute position of the i-th anchor, and $\bm{p}_{T,j}\in\mathbb{R}^3$ the absolute position of the j-th tag on the rover. These ranging measurements are assumed to be obtained through UWB technology. Thus, we will refer to them both as ranging and UWB measurements. Lastly, the quaternion dynamics \cite{Challa} depend on the antysimmetric matrix

\begin{equation}
	\label{prostat:eqn:OmegaMat}
	\bm{\Omega}(\bm{\xi}) =
	\begin{bmatrix}
		-[\bm{\xi}\times] & \bm{\xi} \\
		-\bm{\xi}^T       & 0
	\end{bmatrix} \quad \in \mathbb{R}^{4\times 4},
\end{equation}

\medskip
with

\begin{equation}
	\label{probstat:eqn:omegaCross}
	[\bm{\xi}\times] =
	\begin{bmatrix}
		0      & -\xi_3 & \xi_2  \\
		\xi_3  & 0      & -\xi_1 \\
		-\xi_2 & \xi_1  & 0
	\end{bmatrix} \quad \in \mathbb{R}^{3\times 3}.
\end{equation}

\medskip
As in \cite{Oliva03}, we consider output $\bm{y}$ to be sampled at different rates for the IMU and UWB measurements, namely, we assume them to be available every $\Delta t_a$ and $\Delta t_d$, respectively. More clearly, it holds

\begin{subequations}
	\begin{align}
		\label{probstat:eqn:deltatimes}
		\Delta t_a & = t_{a,q} - t_{a,q-1}, \\
		\Delta t_d & = t_{d,l} - t_{d,l-1},
	\end{align}
\end{subequations}

\medskip
with $(q,l)\in\mathbb{N}$, $\Delta t_a < \Delta t_d$, and $\Delta t_d = h\cdot\Delta t_a$ with $h\in\mathbb{N}_0\gg 1$. Indeed, this assumption renders the UWB dynamics as discrete if compared to the IMU, hence the considered hybrid framework.

\subsection{Hybrid systems}
\label{probstat:hybsys}

We now recall shortly the fundamentals of hybrid systems, as introduced in \cite{GOEBEL2009}. Hybrid systems are characterized by a continuous dynamic described by a set of ODEs (i.e., flow-map), and a discrete dynamic, described by a set of difference equations (i.e., jump-map):

\begin{subequations}
	\begin{equation}
		\mathcal{P}_h :
		\begin{aligned}
			  &
			\begin{cases}
				\dot{\bm{\chi}} & = \tilde{A}\bm{\chi} + \tilde{B}\bm{\mu} \\
				\dot{\tau}      & = 1
			\end{cases}
			, & \ (\bm{\chi},\tau)\in\mathcal{C}, \\
			  &
			\begin{cases}
				\bm{\chi}^+ & = A^+\bm{\chi} + B^+\bm{\mu} \\
				\tau^+      & = 0
			\end{cases}
			, & \ (\bm{\chi},\tau)\in\mathcal{D},
		\end{aligned}
	\end{equation}
	\label{probstat:eqn:plant_dyn_hybrid}
\end{subequations}

\medskip
where $\chi\in\mathbb{R}^n$ is the state, $\mu\in\mathbb{R}^m$ is the input, $\mathcal{C}$ is the flow-map domain and $\mathcal{D}$ is the jump-map domain. In our specific case, $\bm{\chi} = \bm{x}$, $\bm{\mu} = \bm{u}$, $\mathcal{C} \triangleq \{\bm{x}\in\mathbb{R}^n, \ \tau \in [0,\Delta t_d)\}$, $\mathcal{D} \triangleq \{\bm{x}\in\mathbb{R}^n, \ \tau\geq\Delta t_d\}$, with $n=16$, and $m=6$. Note that the time variable $\tau$ keeps track of the time flow, and univoquely triggers the jump-map reset events. Indeed, a wide number of plants can be described within this framework. Also, appropriate stability analysis should be considered when dealing with this kind of systems. For a comprehensive analysis of this topic, the reader may refer to \cite{GOEBEL2009}. Indeed, in our case, the idea is to embody the sensor dynamics of the plant and observer within the flow-map, exploiting IMU-related correction terms in $\mathcal{C}$, and the UWB-related ones in $\mathcal{D}$.

\subsection{Outline}
\label{probstat:outline}

This work provides two main contributions; the first consists in extending the hybrid observer proposed in \cite{Oliva03} to estimate also the orientation. The design procedure exploits the \textit{Trajectory Based Optimization Design} method, proving its efficiency also for more complex systems and tasks. Furthermore, it succeds in bringing the entire design process from a simulated implementatio to a real world scenario. More specifically, the proposed solution is tested on a Jackal rover from Clearpath. The control architecture will be detailed later on in the paper but it has been implemented in ROS Melodic, exploiting standard IMU sensors and Decawave UWB technology \cite{UWBdecawave}.

%This work addresses the localization of a terrestrial rover moving in a 3D environment. More specifically our goal is to estimate position and orientation of the rover in a GPS-denied environment, exploiting only IMU and ranging measurements. Ranging measurements are provided through low-cost UWB sensors, and are referenced from a four anchors mesh deployed on site.

%\medskip
%We are particularly interested in reaching a good precision in the estimation on the z-axis. Indeed, as previously highlighted in [CDC], standard EKF-based solutions fail to meet this requirement as the sensitivity of the measurement with respect to the heightis higher due to the nonlinearities in the output mapping structure.

%\medskip


%\medskip
%we now proceed by detailing the system dynamics and the measurement setup.

